%%%%%%%%%%%%%%%%%%%%%%%%%%%%%%%%%%%%%%%%%%%%%%%%%%%%%%%%%%%%%%%%%%%%%%
% How to use writeLaTeX: 
%
% You edit the source code here on the left, and the preview on the
% right shows you the result within a few seconds.
%
% Bookmark this page and share the URL with your co-authors. They can
% edit at the same time!
%
% You can upload figures, bibliographies, custom classes and
% styles using the files menu.
%
%%%%%%%%%%%%%%%%%%%%%%%%%%%%%%%%%%%%%%%%%%%%%%%%%%%%%%%%%%%%%%%%%%%%%%

\documentclass[12pt]{article}
\usepackage{adjustbox}
\usepackage{sbc-template}
\usepackage{todonotes}
\usepackage{graphicx,url}
\usepackage{amsmath}
\usepackage{multirow}
\usepackage[utf8]{inputenc}  
\usepackage{babel}
\usepackage[T1]{fontenc}
\usepackage{xspace}
\usepackage{url}
\usepackage{graphicx}
\usepackage{subfig}
%----------
\usepackage{unicode-math}
\usepackage{babel}
\babeltags{br = brazil, en = english}
\usepackage[T1]{fontenc}
\usepackage{xspace}
\usepackage{url}
\usepackage{lipsum} 
%\setmathfont{xits-math.otf}
\usepackage{enumerate}
%\setmathfont[math-style=upright,range={`e,`i}]{xits-math.otf}

%--------------
\sloppy

\title{Calculo Numérico\\Trabalho\_4}


\author{Prof. Dra. Larissa de Freitas \inst{1},\\Guilherme de Souza\inst{1}}

\begin{document} 

\maketitle
\br
%\section{Funções}
%\begin{eqnarray}
%$\lambda x: 5x^{3} - 2x^{2} + 8x - 10$\\
%$\lambda x: 2x^{3} + 5x^{2} + \sin x - 30$\\
%$\lambda x: e^{-x2}\cos x$\\
%    $\lambda x: (x+1)(x-1)(x-3)^{5}\\
%    $\lambda x: (x+2)^{3}\sqrt{x^{2}+1}$
%\end{eqnarray}

\section{Métodos implementados}
\begin{itemize}
  \item Trapezio;
  \item 1/3 de Simpsom;
  \item 3/8 de Simpsom ;
  \item Euler;
  \item Runge-Kutta 2a Ordem;
  \item Runge-Kutta 4a Ordem;
\end{itemize}

Para execução de tais métodos como, \textbf{Trapezio, 1/3 de Simpsom e 3/8 de Simpsom} foram aplicados na lista de exercícios 11\footnote{Disponivel no AVA:\url{https://ava.ufpel.edu.br/pre/pluginfile.php/318455/mod_resource/content/1/ListaDeExercicios11.pdf}}. Os métodos de \textbf{Euler}, \textbf{Runge-Kutta 2a ordem} e \textbf{4a ordem} utilizou-se a lista de exercícios 12\footnote{Disponivel no AVA:\url{https://ava.ufpel.edu.br/pre/pluginfile.php/320366/mod_resource/content/2/ListaDeExercicios12.pdf}}.


Todos os métodos foram desenvolvido na linguagem de programação python, fazendo uso da plataforma Colaboratory (plataforma do Google), o código se encontra disponivel na plataforma para acesso via link\footnote{Código:\url{https://colab.research.google.com/drive/1D2WZN_HYdlaP5LdSa-c4y9_G4DBMhnHT}}

\section{Método do Trapezio}
 O método dos trapazeio é um metodo numerico para aproximação da resolução de um integral definida.
 \[\int_{a}^{b} \! fx \, dx = \]

 Dado que no calculo da integral temos diversos procedimentos para resolução desta integral, seja por substituição, integral por partes, substituição trigonométrica, entre outras. Mas o Método do trapézio por sua vez, nos retorna um valor aproximado independendo do método utilizado.

\begin{figure}[ht]
    \centering
    \includegraphics[scale=0.3]{/home/souza/Documents/semestre_2019-2/calculo_numerico/trabalho_4/img/trapezio2.png}
    \caption{Exemplo de de atuação do método dos trapézios}
\end{figure}

Sabemos que a area a limitada pelos limites da integral, como demonstrado na figura 1, o que o método se propoe a realizar a divisão desta area em alguns trapézios ao final realiza sua soma, dessa forma obtendo uma aproximação.


 \[\int_{a}^{b} \! fx \, dx = \sum_{k=1}^{n}Ak \]

%---------------------------------------------------------------------------------------------------
% Resultados 
%---------------------------------------------------------------------------------------------------
\begin{itemize}
    \item \textbf{(B)} Trapezio: 31.365285650063754
\end{itemize}

\begin{table}[ht]
\centering
\begin{tabular}{|lllll|}
-1.08268227 & -0.73575888 & 0.0 & 5.43656366 & 59.11244879
\end{tabular}
    \caption{Função utlizando a regra do trapezio no exercício B da lista 11}
\end{table}


\begin{itemize}
    \item \textbf{(C)} Trapezio: 0.7842407666178157
\end{itemize}

\begin{table}[ht]
\centering
\begin{tabular}{|lllllll|}
  0.5 & 0.97297297 & 0.9 & 0.8 & 0.69230769 & 0.59016393 & 0.25
\end{tabular}
    \caption{Função utlizando a regra do trapezio no exercício C da lista 11}
\end{table}


\begin{itemize}
    \item \textbf{(D)} Trapezio: 0.10486282062502501
\end{itemize}

\begin{table}[ht]
\centering
\begin{tabular}{|lllllll|}
  0.0 & 0.20999043 & 0.69856645 & 1.08060461 & 0.83640026 & -0.53179749 & -1.66458735
\end{tabular}
    \caption{Função utlizando a regra do trapezio no exercício D da lista 11}
\end{table}


\begin{itemize}
    \item \textbf{(E)} Trapezio: -13.575979391799388
\end{itemize}
\begin{table}[ht]
\centering
\begin{tabular}{|lllllllll|}
0.0 & 2.24766465 & 5.42294499 & 6.97416428 & 2.08548731 & -13.92608463 & -39.26843454 & -56.88800791 & -15.25556929
\end{tabular}
    \caption{Função utlizando a regra do trapezio no exercício E da lista 11}
\end{table}


\begin{itemize}
    \item \textbf{(F)} Trapezio: 0.5196110146984233
\end{itemize}
\begin{table}[ht]
\centering
\begin{tabular}{|lllllllll|}
  0.4  & 0.71910112  & 0.64 & 0.56637168 & 0.5 & 0.44137931 & 0.3902439 & 0.34594595 & 0.15384615
\end{tabular}
    \caption{Função utlizando a regra do trapezio no exercício F da lista 11}
\end{table}


\section{Simpsom}

Dando continuidade com métodos de integração, dado a dificuldade de se resolver a integral partindo do principio que muitas vezes se quer há o conhecimento da função, faremos uso do método de Simpsom.

\begin{figure}[!h]
    \centering
    \includegraphics[scale=0.5]{/home/souza/Documents/semestre_2019-2/calculo_numerico/trabalho_4/img/simpson.jpeg}
    \caption{Exemplo de atuação do método de Simpson}
\end{figure}


Dado o exemplo da figura 2, pegamos os dois limites da integral conhecidos e divido em partes suas partes iguais de maneira a encontrar o $h$, como a seguir:

$h=\frac{x_{k+1} - x_{k-1}}{2}$

Dessa forma terei mais pontos sobre o grafico da função, assim interpolando estes pontos, ou seja, procuro qual é o polinomio de menos grau possivel que passe sobre todos, assim calculando a area entre o grafico do polinomio e o intervalo dos eixos do x, esta area é uma aproximação da função f da integral ou seja:


 \[\int_{k-1}^{k+1} \! fx \, dx \approx  \int_{k-1}^{k+1} \! P_{2}x \, dx \]

Pois dessa forma o polinomio é facil de integrar, assim é facil chegar a aproximação da integral. Neste trabalho fizemos uso de dois de seus métodos conhecidos como 1/3 de Simpson, para polinomios de segunda ordem, e 3/8 de Simpson para polinomios de ordem superiores, (a demonstração será ocultada pela vasta demonstração disponvel na internet e pela sua extensa explicação).

Desta forma tais métodos se propoe a serem mais precisos que o métodos do trapézio, geralmente, nem sempre cumprido tal afirmação dada determinada integral e o polnomio.

\subsection{Aplicação de 1/3 de Simpson}
%---------------------------------------------------------------------------------------------------
% Resultados
%---------------------------------------------------------------------------------------------------
\begin{itemize}
    \item \textbf{(B)} 1/3 de Simpsom: 22.231872397453277
\end{itemize}
\begin{table}[ht]
\centering
\begin{tabular}{|lllll|}
 -1.08268227 & -1.47151776 & -0.73575888 & 10.87312731 & 10.87312731
\end{tabular}
    \caption{Função utlizando a regra do trapezio no exercício B da lista 11}
\end{table}


\begin{itemize}
    \item \textbf{(C)} 1/3 de Simpsom: 0.8054718653079309
\end{itemize}
\begin{table}[ht]
\centering
\begin{tabular}{|lllllll|}
     0.5 & 1.94594595 &  0.97297297 & 1.6 & 0.8 & 1.18032787 & 0.29508197
\end{tabular}
    \caption{Função utlizando a regra do trapezio no exercício C da lista 11}
\end{table}


\begin{itemize}
    \item \textbf{(D)} 1/3 de Simpsom: 0.12706697873621364
\end{itemize}
\begin{table}[ht]
\centering
\begin{tabular}{|lllllll|}
     0.0 & 0.41998087 &  0.20999043 & 2.16120922 & 1.08060461 & -1.06359498 & -0.26589874
\end{tabular}
\caption{Função utlizando a regra do trapezio no exercício D da lista 11}
\end{table}


\begin{itemize}
    \item \textbf{(E)} 1/3 de Simpsom: -11.92869601630686
\end{itemize}
\begin{table}[ht]
\centering
\begin{tabular}{|lllllllll|}
    0.0 & 4.49532929  &  2.24766465 & 13.94832857 & 6.97416428 & -27.85216926 & -13.92608463 & -113.77601581  & -28.44400395
\end{tabular}
    \caption{Função utlizando a regra do trapezio no exercício E da lista 11}
\end{table}


\begin{itemize}
\item \textbf(F)} 1/3 de Simpsom: 0.5355245326505693
\end{itemize}
\begin{table}[ht]
\centering
\begin{tabular}{|lllllllll|}
    0.4 & 1.43820225  & 0.71910112 & 1.13274336 & 0.56637168 & 0.88275862 & 0.44137931 & 0.69189189  & 0.17297297
\end{tabular}
    \caption{Função utlizando a regra do trapezio no exercício F da lista 11}
\end{table}


\subsection{Aplicação de 3/8 de Simpsom}


%--------------------------------------------------------------------------------------------------------------------------------
%    Resultados
%--------------------------------------------------------------------------------------------------------------------------------
\begin{itemize}
\item \textbf(B)} 3/8 de Simpsom: 24.405365132780666
\end{itemize}

\begin{table}[ht]
\centering
\begin{tabular}{|lllll|}
    -1.08268227 & -1.10363832 & 0.0 & 8.15484549 & 59.11244879 
\end{tabular}
    \caption{Função utlizando a regra do trapezio no exercício B da lista 11}
\end{table}


\begin{itemize}
\item \textbf(C)} 3/8 de Simpsom: 0.7552413857741727
\end{itemize}
\begin{table}[ht]
\centering
\begin{tabular}{|lllllll|}
    0.5 & 1.45945946 & 1.35 & 1.2 & 0.69230769 & 0.59016393 & 0.25
\end{tabular}
    \caption{Função utlizando a regra do trapezio no exercício C da lista 11}
\end{table}


\begin{itemize}
\item \textbf(D)} 3/8 de Simpsom: 0.2029697091109325
\end{itemize}
\begin{table}[ht]
\centering
\begin{tabular}{|lllllll|}
    0.0 & 0.31498565 & 1.04784968 & 1.62090692 & 0.83640026 & -0.53179749& -1.66458735
\end{tabular}
    \caption{Função utlizando a regra do trapezio no exercício D da lista 11}
\end{table}


\begin{itemize}
\item \textbf(E)} 3/8 de Simpsom: -11.336218635489457
\end{itemize}
\begin{table}[ht]
\centering
\begin{tabular}{|lllllllll|}
    0.0 & 3.37149697 & 8.13441749 & 10.46124643 & 2.08548731 & -13.92608463 & -58.90265182 & -56.88800791 & -15.25556929
\end{tabular}
    \caption{Função utlizando a regra do trapezio no exercício E da lista 11}
\end{table}


\begin{itemize}
\item \textbf(F)} 3/8 de Simpsom: 0.4982574816855577
\end{itemize}
\begin{table}[ht]
\centering
\begin{tabular}{|lllllllll|}
    0.4 & 1.07865169 & 0.96 & 0.84955752 & 0.5 & 0.44137931 & 0.58536585 & 0.34594595 & 0.15384615
\end{tabular}
    \caption{Função utlizando a regra do trapezio no exercício E da lista 11}
\end{table}

\section{Euler}


Como os demais metodos aqui apresentados, o método foca em encontar aproximações de forma a resolver a $fx$. Com tal método é possivel partindo de um ponto conhecido, como, $x_{0}$ $y_{0}$, com isso consigo andar um passo no grafico usando a inclinação da tangente e da secante, ou seja:

$\left \{ \begin{matrix} y' = f(x,y) & \mbox{ }\mbox{ }\\ y(x_{0}) = y_{0} & \mbox{ }\mbox{} \end{matrix} \right.$ 


$h = x_{1} - x_{0}$

$\frac{y_{1} - y_{0}}{h} \approx y'(x_{0})$

$y_{1} - y_{0} \approx hy'(x_{0})$

$y_{1} \approx y_{0} + hy'(x_{0})$

Com isso. apartir do ponto em que está no grafico, é possivel avançar um tamanho $h$ e encontrar o valor $x_{1}$ com isso é possivel encontrar a imagem, $y_{1}$. Ou seja, os novos valores estão bem proximo do grafico da função, dessa forma, basta repetir o metodo mais vezes, de forma a encontrar mais pontos proximos conforme aumenta em um tamanho $h$, a partir do resultado encontrado anteriormente como pode ser visto na figura 3.

\begin{figure}[!h]
    \centering
    \includegraphics[scale=0.07]{/home/souza/Documents/semestre_2019-2/calculo_numerico/trabalho_4/img/euler.png}
    \caption{Exemplo de atuação do método de Simpson}
\end{figure}

\section{Runge-Kutta}
Dado sua precisão e simplicidade um dos métodos mais utilizados é o método de Runge-kutta. è uma familia de métodos que possuem a precisão dos métodos de Taylor, porém não exige o calculo das derivadas de ordem superior.

A demonstração do método base será ocultada como alguns anteriormente aqui apresentados, dado a sua semelhança na sua primeira ordem ao método de Euler e Taylor de primeira ordem.

\subsection{Runge-Kutta 2a ordem}

Dentre a familia deste método temos o de 2a ordem, que é definido por\footnote{Formula disponivel em:\url{http://conteudo.icmc.usp.br/pessoas/andretta/ensino/aulas/sme0300-2-13-fisica/aula20-edorungekutta.pdf}}:

$\omega_{i+1} = \omega_{i} + h(c_{1}k_{1} + c_{2}k_{2})$

com,

$c_{1}$ + $c_{2}$ = 1,

$k_{1}$ = $f(t_{i}$, $\omega_{i}$),

$k_{2}$ = $f(t_{i}$ + $ha_{a}$, $\omega_{i}$ + $h(a_{2}k_{1})$.


Dessa forma podemos determinar $c_{1}$, $c_{2}$ e $a_{2}$, desenvolvendo a função $k_{2}$ pelo polinomio de Taylor, em torno de ($t_{i}, \omega_{i}$) até segunda ordem. Assim acabamos por obter um sistema não linear que possui diversas soluções.

\subsection{Runge-Kutta 4a Ordem}
Através de alguns calculos e algumas mudanças nas variaves tanto da do Runge-Kutta 2a ordem quanto o Runge-Kutta 4-estagios, conseguiremos defirnir um Método de Runge-Kutta 4a ordem. Através dos calculos chegaremos a um sistema não-linear com solução unica.

\section{Disertando sobre os Resultados}

Ao analisarmos os resultados encontrados a apartir da aplicação dos métodos de 1/3, 3/8 de Simpson e o método dos trapezios na lista de exercicio 11, podemos observar que os resultados são bastantes semelhantes, os mesmo podem ser encontrado em suas devidas seções e nas tabelas de 1 à 15.

Dado que o Método dos trapezios foi o que apresentou resultados levemente distante dos demais por dois ou mais. Como exemplo podemos pegar o exercicio (B) no qual ele apresenta 7 a mais que o 3/8 e 9 comparado ao 1/3, da mesma forma ja demonstra a leve diferença entre o 1/3 e 3/8 de Simpson, sendo somnete por dois e menor ainda ao comparar os demais resultados encontrados. Uma boa forma de analisar e que facilita ver a diferença, é ao observar o os valores antes da soma do vetor.

Nos resultados aqui apresentados, ou seja, nesta determinada situação, as descrições dos métodos se fizeram por valer, onde conseguirmos observar que os métodos se Simpson tendem a serem mais precisos comparado ao método dos trapezios.

% QUESTÃO 4, APLICADO RK4 COM GRAFICO
Na questão 4 da lista 12, ao modificarmos o tamanho do $h$ no Runge-Kutta 4a ordem, podemos notar ao observar o grafico na figura 4, que passos menores aparentemente nos retorna uma presição um pouco melhor, realizando um curva mais aproximativa a qual esperamos. Mas podemos obervar que isso faz com que eu tenha um numero maior de repetição, enquanto em $h$=0.5 realizamos somente 4, com $h$=0.25 realizamos 8 repetições.


\begin{figure}[!h]
    \centering
    \includegraphics[scale=0.7]{/home/souza/Documents/semestre_2019-2/calculo_numerico/trabalho_4/img/rk4Questao4.png}
    \caption{Resultado da aplicação do Metodo Runge-Kutta 4a Ordem na questão 4 da lista 12}
\end{figure}

Mas por outro lado ao observarmos a Tabela 16 e 17, com o $h$ maior aparentemente o resultado tende a convergir mais rapido, de forma que é possivel observar ao compararmos a coluna K4 de ambas as tabelas.

\begin{table}[!h]
    \centering
\begin{tabular}{c|c|c|c|c|c}
\textbf{Xk} & \textbf{Yk} & \textbf{K1} & \textbf{K2} & \textbf{K3} & \textbf{K4} \\ \hline
0.0         & 1.000000    & -1.000000   & -0.703125   & -0.772705   & -0.460236   \\
0.5         & 0.632342    & -0.474257   & -0.224778   & -0.252065   & 0.000000    \\
1.0         & 0.513347    & 0.000000    & 0.288758    & 0.329364    & 0.847536    \\
1.5         & 0.686995    & 0.858744    & 1.859718    & 2.375845    & 5.624753    \\ \hline
\end{tabular}
    \caption{Resultado da questão 4 da lista 12 com $h$=0.5 usando o método de Runge-Kutta 4a ordem}
\end{table}


\begin{table}[!h]
    \centering
\begin{tabular}{c|c|c|c|c|c}
\textbf{Xk}               & \textbf{Yk}                   & \textbf{K1}                   & \textbf{K2}                   & \textbf{K3}                   & \textbf{K4}                  \\ \hline
0.0                       & 1.000000                      & -1.000000                     & -0.861328                     & -0.878391                     & -0.731627                    \\
0.25                      & 0.782872                      & -0.733943                     & -0.593939                     & -0.608979                     & -0.472971                    \\
0.50                      & 0.632341                      & -0.474256                     & -0.349208                     & -0.358733                     & -0.237413                    \\
0.75                      & 0.543693                      & -0.237866                     & -0.120459                     & -0.123899                     & 0.000000                     \\
\multicolumn{1}{l|}{1.0}  & \multicolumn{1}{l|}{0.513419} & \multicolumn{1}{l|}{0.000000} & \multicolumn{1}{l|}{0.136377} & \multicolumn{1}{l|}{0.140905} & \multicolumn{1}{l}{0.308613} \\
\multicolumn{1}{l|}{1.25} & \multicolumn{1}{l|}{0.549385} & \multicolumn{1}{l|}{0.309029} & \multicolumn{1}{l|}{0.523699} & \multicolumn{1}{l|}{0.547598} & \multicolumn{1}{l}{0.857855} \\
\multicolumn{1}{l|}{1.50} & \multicolumn{1}{l|}{0.687279} & \multicolumn{1}{l|}{0.859099} & \multicolumn{1}{l|}{1.303750} & \multicolumn{1}{l|}{1.394939} & \multicolumn{1}{l}{2.136779} \\
\multicolumn{1}{l|}{1.75} & \multicolumn{1}{l|}{1.036998} & \multicolumn{1}{l|}{2.138809} & \multicolumn{1}{l|}{3.281255} & \multicolumn{1}{l|}{3.640500} & \multicolumn{1}{l}{5.841370} \\ \hline
\end{tabular}
    \caption{Resultado da questão 4 da lista 12 com $h$=0.25 usando o método de Runge-Kutta 4a ordem}
\end{table}

Semelhantemente temos a questão 3 da lista de exercicio 2, porém o método de aplicação utilizado é o método de Euler, sofrendo algumas leves mudança pela forma que foi emplementado o código, onde podemos obervar ser realizado mais uma repetição que o método anterior aplicado na questão 4. 


\begin{table}[!h]
    \centering
\begin{tabular}{c|c|c}
\textbf{Xk}              & \textbf{Yk}                   & \textbf{Função}              \\ \hline
0.0                      & 1.000000                      & 0.000000                     \\
0.5                      & 0.500000                      & -0.375000                    \\
1.0                      & 0.312500                      & 0.000000                     \\
0.75                     & 0.312500                      & 0.390625                     \\
\multicolumn{1}{l|}{1.0} & \multicolumn{1}{l|}{0.507812} & \multicolumn{1}{l}{1.523438} \\ \hline
\end{tabular}
    \caption{Resultado da Questão 3 da lista 12 com $h$=0.5 usando o método de Euler}
\end{table}



\begin{table}[!h]
    \centering
\begin{tabular}{l|l|l}
\multicolumn{1}{c|}{\textbf{Xk}} & \multicolumn{1}{c|}{\textbf{Yk}} & \multicolumn{1}{c}{\textbf{Função}} \\ \hline
\multicolumn{1}{c|}{0.0}         & \multicolumn{1}{c|}{1.000000}    & \multicolumn{1}{c}{0.000000}        \\
\multicolumn{1}{c|}{0.25}        & \multicolumn{1}{c|}{0.750000}    & \multicolumn{1}{c}{-0.703125}       \\
\multicolumn{1}{c|}{0.50}        & \multicolumn{1}{c|}{0.574219}    & \multicolumn{1}{c}{-0.430664}       \\
\multicolumn{1}{c|}{0.75}        & \multicolumn{1}{c|}{0.466553}    & \multicolumn{1}{c}{-0.204117}       \\
1.0                              & 0.415524                         & 0.000000                            \\
1.25                             & 0.415524                         & 0.233732                            \\
1.50                             & 0.473957                         & 0.592446                            \\
1.75                             & 0.622068                         & 1.283015                            \\
2.0                              & 0.942822                         & 2.828465                            \\ \hline
\end{tabular}
    \caption{Resultado da questão 3 da lista 12 com $h$=0.25, usando o método Euler}
\end{table}

Ao obersevarmos as Tabelas 18 e 19, podemos notar que quanto menor o $h$ a distância de presição passa a ser 0.3 ou mais, porém os restante o se asemelha as afirmações anterior ditas para o método de Runge-Kutta 4a ordem. Ao analisarmos o Grafico 5 podemos notar que seu comportamento também acaba por se asemlhar ao anterior, porém um disntância maior é apresentada entre as retas dispostas no gráfico.


\begin{figure}[!h]
    \centering
    \includegraphics[scale=0.7]{/home/souza/Documents/semestre_2019-2/calculo_numerico/trabalho_4/img/eulerQuestao3.png}
    \caption{Resultado da aplicação do Metodo de Euler na questão 3 da lista 12}
\end{figure}


%Falando sobre a questão 1 onde os 3 métodos são aplicados

\begin{table}[!h]
    \centering
\begin{tabular}{c|c|c}
\textbf{Xk} & \textbf{Yk} & \textbf{Função} \\ \hline
0.0         & 10.000000   & 0.000000        \\
1.0         & 27.826087   & 38.819758       \\
2.0         & 66.645845   & 36.733675       \\
3.0         & 103.379521  & -25.574117      \\ \hline
\end{tabular}
    \caption{Resultado da questão 1 da lista 12 usando o método de Euler}
\end{table}


\begin{table}[!h]
    \centering
\begin{tabular}{c|c|c|c|c|c}
\textbf{Xk} & \textbf{Yk} & \textbf{K1} & \textbf{K2} & \textbf{K3} & \textbf{K4} \\ \hline
0.0         & 10.000000   & 17.826087   & 30.049930   & 36.435824   & 45.995871   \\
1.0         & 42.798911   & 45.777240   & 37.573938   & 40.719138   & 15.399913   \\
2.0         & 79.092795   & 22.192759   & 3.550366    & 19.570089   & -14.290856  \\ \hline
\end{tabular}
    \caption{Resultado da questão 1 da lista 12 usando Runge-Kutta 4a ordem}
\end{table}


\begin{table}[!h]
    \centering
\begin{tabular}{c|c|c|c|}
\textbf{Xk} & \textbf{Yk} & \textbf{K1} & \textbf{K2} \\ \hline
0.0         & 10.000000   & 17.826087   & -27.826087  \\
1.0         & 5.000000    & 9.456522    & -28.913043  \\
2.0         & -4.728261   & -9.942532   & 44.012377   \\ \hline
\end{tabular}
    \caption{Resultado da questão 1 da lista 12 usando Runge-Kutta 2a ordem}
\end{table}


Para resolução da questão 2 da lista 12, fez-se uso dos métodos Runge-Kutta 2a e 4a ordem, para resolução da questão, foi definida por ela um $h$=0.5 minutos. Podemos obervar nas figuras 6 e 7 onde podemos observar que os metodos obtiveram conclusões semelhantes em 54.0 minutos, podendo observar o comportamento da tabela apresentada na figura, uma parte da mesma foi ocultada para proporcionar uma melhor vizualização.


\begin{figure}[!h]
    \centering
    \includegraphics[scale=2.5]{/home/souza/Documents/semestre_2019-2/calculo_numerico/trabalho_4/img/rk2Questao2.png}
    \caption{Resultado da aplicação do Metodo de Runge-Kutta 2a ordem na questão 2 da lista 12}
\end{figure}



\begin{figure}[!h]
    \centering
    \includegraphics[scale=2.5]{/home/souza/Documents/semestre_2019-2/calculo_numerico/trabalho_4/img/rk4Questao2.png}
    \caption{Resultado da aplicação do Metodo de Runge-Kutta 4a ordem na questão 2 da lista 12}
\end{figure}

\end{document}
